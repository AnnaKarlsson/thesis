\chapter{\textbf{Doing! }CONCLUSIONS}\label{cha:conculsions}
I detta kapitel ska en återkoppling till syftet och frågeställningen/arna ske. Har syftet
uppnåtts och vad blev svaret på frågeställningarna? Här ska också arbetets konsekvenser
för berörd målgrupp och eventuellt för forskare och praktiker beskrivas.

\subsection{\~Acc}
Comp to p.11 in thesis!
Scalability.\\
Influence of the version of Operating Systems \\
As you also can see in the~\figureref{fig:iPhoneScatter} is that the iOS has an inverse z-axis compared to the Android devices in~\figureref{fig:scatter-withGrav}. This makes this is a good feature in authentication purposes since ...
Coolt att mean, min och max values är exkt samma fast än RMS ändras
Time/permanence koppla med \cite{sensor:accelPrint} om att OS inte påverkar.

\subsection{\~Gyro} 
Comparing the result of the measurements with the seven characteristics a good biometric trait should have (\sectionref{auth:bio:character}), it fails in the . That is , uniqueness and permanence. Measurability and  performance is good compared to 

Another thing to notice about the result is that dealing with pictures is much more time 

\section{\textbf{ToDo! }Further work}
Man bör också ha ett stycke om framtida arbete där man beskriver vad man skulle vilja
göra om man hade mer tid eller som rekommendationer för framtida studier eller exjobb.
Under arbetets gång har du säkert fått flera egna idéer om förbättringar. Det är viktigt att
du tar fram sådana synpunkter. Om man har ett sådant stycke är det dock viktigt att det är
konkreta och väl genomtänkta förslag som presenteras, snarare än vaga idéer. 