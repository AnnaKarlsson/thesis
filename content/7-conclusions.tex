\chapter{\textbf{ToDo! }CONCLUSIONS}\label{cha:conculsions}
Finns det något i resultaten som står ut och behöver analyseras och kommenteras? Hur
förhåller sig resultaten till det material som togs upp i teorigenomgången? Vad säger teorin
om vad resultaten egentligen betyder? Vad innebär det till exempel att man vid en
användbarhetsmätning av ett nytt system fått ett visst värde; hur bra eller dåligt är det?
Finns det något i resultaten som är oväntat baserat på teorigenomgången, eller stämmer
det bra överens med vad man teoretiskt kunde förvänta sig?


\section{Ethical aspects}\label{sec:ethical}
TODO!

\section{Conclusions of measurements method}
Här ska den använda metoden diskuteras och kritiseras. Att ha ett kritiskt förhållningssätt
till använd metod är en viktig del av vetenskaplighet.
En studie är sällan perfekt. Det finns nästan alltid saker man skulle vilja gjort annorlunda
om man kunnat göra om studien eller haft extra resurser. Gå igenom de viktigaste
bristerna du ser med din metod och diskutera tänkbara konsekvenser för resultaten.
Koppla tillbaka till den metodteori som togs upp i teorikapitlet. Referera explicit till
relevanta källor.
Diskussionen ska också visa en medvetenhet om metodologiska begrepp såsom
replikerbarhet, reliabilitet och validitet. Replikerbarhet har redan tagits upp i stycket om
metod. Reliabilitet är ett begrepp för huruvida man kan förvänta sig att få samma resultat
om man gör om en studie med samma metod. En studie med hög reliabilitet har en hög
sannolikhet av att kunna upprepas med samma resultat. Validitet handlar lite förenklat om
huruvida man i en mätning mätt det man tror sig mäta. En studie med hög validitet har
alltså en hög grad av trovärdighet. Dessa termer måste mappas över till det aktuella
sammanhanget och diskuteras.
Metoddiskussionen ska också innehålla ett stycke om källkritik. Här diskuteras författarnas
förhållningssätt till källor och vilka avvägningar som gjorts.\\
\\
I vissa sammanhang kan det vara så att den mest relevanta informationen för studien inte
finns i vetenskaplig litteratur utan hos enskilda programutvecklare och i open-sourceprojekt.
Det måste då tydliggöras att ansträngningar gjorts för att ta del av denna
information, till exempel via direktkontakter med utvecklare och diskussioner på forum,
osv. Likaså måste ansträngningar ha gjorts för att faktiskt visa avsaknaden av relevant
vetenskaplig litteratur. Exakt hur dessa ansträngningar gjorts redovisas lämpligen i ett
metodstycke. Källkritikstycket i diskussionskapitlet ska kritiskt granska denna metod.
\subsection{Motion sensors}
Scalability.\\
Influence of the version of Operating Systems \\

\subsection{Camera sensor}


\section{Conclusions of result}\label{sec:conclusion}
I detta kapitel ska en återkoppling till syfte och frågeställningar ske. Har syftet uppnåtts
och vad blev svaret på frågeställningarna? Här ska också arbetets konsekvenser för
berörd målgrupp och eventuellt för forskare och praktiker beskrivas.
Man bör också ha ett stycke om framtida arbete där man beskriver vad man skulle vilja
göra om man hade mer tid eller som rekommendationer för framtida studier eller exjobb.
Under arbetets gång har du säkert fått flera egna idéer om förbättringar. Det är viktigt att
du tar fram sådana synpunkter. Om man har ett sådant stycke är det dock viktigt att det är
konkreta och väl genomtänkta förslag som presenteras, snarare än vaga idéer.
\subsection{Accelerometer}
\subsection{Gyroscope}
\subsection{Camera}
\subsection{Implementation}

\section{Further work}
Det ska ingå ett stycke med en diskussion om etiska och samhälleliga aspekter relaterade
till arbetet. Detta är viktigt för att påvisa professionell mognad samt för att utbildningsmålen
ska kunna uppnås. Om arbetet av någon anledning helt saknar koppling till etiska eller
samhälleliga aspekter ska detta explicit anges i stycket Avgränsningar i inledningskapitlet.
I diskussionskapitlet ska man explicit referera till källor som är relevanta för diskussionen.