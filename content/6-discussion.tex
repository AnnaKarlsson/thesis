\chapter{DISCUSSION}\label{cha:discussion}
This chapter interweave the theory and method with the result. What the different is between the theory and result and why.The

\section{\textbf{ToDo! }Accelerometer}
\subsection{Result}
As written in~\cite{acc:kionixerr} the sources of bias in an accelerometer has to be compensated for. The manufacturer of the mobile devices makes some of this compensation before JavaScript gets the data. The results of the ~\figureref{fig:scatter-withoutGrav} and~\figureref{fig:scatter-withGrav} indicates that JavaScript also makes compensating to the output of the event \texttt{acceleration}, probably to ease the development of software using the accelerometer, e.g. games. This however is not anything that is public in any specifications such~\cite{sensor:W3Cspec} or~\cite{sensor:accIncludingGravity}. The Android developer page about sensor event \cite{android:sensorEvent} state that they make factory calibration and temperature compensation even on their uncalibrated sensor events of (only magnetometer and gyroscope) that is relativity new feature added in Android 4.3 Jelly Bean (API level 18~\cite{android:API18}) from 2013 but the original once used since Android 1.5 Cupcake (API level 3~\cite{android:API3}) from 2009 makes some more noise compensation and calibration. What kind of compensation and calibration done is not public. 

The first pre-measurements of the accelerometer seemed promising and the fact that you easily could see the differences between devices of the same brand and model. 
\subsection{Method}

\section{Gyroscope}
Discussion about the result of the gyroscope measurements and the method used to get gyroscope data from the mobile device.
\subsection{Result}
The first method used to compare the measurements were based on research of the accelerometer since there were no earlier research made on the gyroscope. This may effect the outcome since there may be other features that would have given better result. \\
The other method by calculating the Allan variance that is used for calibration of gyroscope may didn't gave that result that were expected. Since the variance often is used for gyroscope calibration it may be the case that it already is calculated and compensated for in the device.\\
\\
The gyroscope seems to much more sensitive in measurements than the accelerometer and therefore be harder to extract the constant bias. The fact that Android or JavaScript doesn't reveal information on what bias compensation that has been done before the developer get the measurement data makes that part harder. That the gyroscope is much more sensitive than the accelerometer can be seen when reading from table~\ref{featureGyro} were the \textit{Sony Xperia Z1 Compact} device changed the min median distance with 75\% and the \textit{Google Nexus 7} only with 6\%. As the \textit{Sony Xperia} has been used over the fifty days of measurements. Compared to the \textit{Nexus} that only were used at the time of the measurements, thus didn't get dropped or else that could caused affected the hardware part of the device.
\subsection{Method}
The method using JavaScripts listener to collect the data seems to work as expected. The question to ask is the same as for the accelerometer how much calibration and compensation of bias and drift already done before the software developers gets the output from the gyroscope. The positive thing about using JavaScript instead of developing an application is that the diversity of the collected devices is much better. It also gets easier to collect measurements since it is a web-page is much easier to spread and no installation is needed, in context to an application that has to bee installed. The gain of using an Android application when measuring the gyroscope would be that Android provide an uncalibrated version of the gyroscope since 2013~\cite{android:API18}. This rawer data may result in better feature values in time domain or Allan variance.


\section{Camera}
This section discuss the result and method used for evaluate the camera sensor as a fingerprinting characteristic.
\subsection{Result}
The result of the camera sensor weren't as good as expected or as good as in the research by ~\cite[]{sensor:camera:DCIdent} were PRNU also was used. The significant differences is the use of a mobile device camera instead of a digital camera. Although the high level specification seems to bee comparable with the digital cameras from 2009, since they had around 11 mega-pixels, an images size of around 4000x3000 pixels, and digital zoom of 4 times and had HD video recording width 30 fps.~\cite[]{gbg:kamera} This is about the average smart-phones camera today, but some other specifications may have other impact as ISO, optical zoom etc. \\

\subsection{Method}
The two methods used for collecting picture features had different advantages and disadvantages. The video-collecting done in connection to the second measurement were good in terms of measurability since it was easy to record a video of five seconds and just send. It generated in 100-200 which also made a enough pictures for a trait. On the other hand that lead to worse result in terms of uniqueness. \\
The second way of collecting data weren't as good in terms of measurability but it god slightly more uniqueness but far from good enough.


\section{The work in a wider context}\label{sec:ethical}
There is a lot of discuss in terms of privacy and integrity when dealing with the sensor of the device. To begin with neither of the motion sensors require any permission to read when visiting a web-page. If there is an easy way of identifying a device by a sensor the days of using cookies will be long gone. Which of course can have advantage in terms of user-ability, but as valuable your personal information is today for the commercial and advertising its hard to set a value for something that could identify you everywhere on the Internet. The tracking possibilities is enormous and has to be concerned if this type of identifying can be done. \\
\\
There are of course some good things in the view of ethical and societal aspects. If the sensor-data is used as aimed in this thesis it gain  privacy and integrity since the possibility of more secure authentication both between human and machine and M2M. Because, you want such that it's really any of your heat sensors that send signals to your thermostat, or that it is only your mobile that can unlock the front door.\\
\\
The point here is that fingerprinting features of a device should be treated in the same way as your biometric trait. This means that if you must have control over were the biometric trait is used. Most of us think that it is legit that Authority used our fingerprint if it gain in a more secure society. On the other hand most of us don't want our fingerprints to be used in commercial purposes. \\
The concept should be considered when fingerprinting a device as well. The accelerometer data may be applicable to used by banks, to your door or car. But you may not want is as a login feature to a commercial site that may sell that information.


