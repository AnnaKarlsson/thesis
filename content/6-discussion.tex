\chapter{DISCUSSION}\label{cha:discussion}
This chapter interweaves the theory and method with the result. What the difference between the theory and result is and why. The limitations of the method used is also discussed. \\
Discussion followed by conclusion in the view of a biometric system can be seen as~\textit{choose features and matching algorithm} and also~\textit{choosing feature and matching algorithm}.

\section{Accelerometer}
\subsection{Result}
The result of the first measurements of the accelerometer resulted in some unexpected result, the fact that JavaScripts listener without gravity does not seem to have any static noise at all. The reason could be some software modification of the sensor data before it reaches the \texttt{event}. The recommendation from MEMS accelerometer manufacturers is to calibrate the sensors.~\cite[]{acc:kionixerr}\\ 
Doing some research on Android sensors revield that their \texttt{SensorEvent} also have two types of accelerometer sensors that can be used: \\
\texttt{TYPE\_ACCELEROMETER} is the hardware measurements that measures the force of acceleration including the force of gravity with the SI unit $m/s^2$. \\
\texttt{TYPE\_LINEAR\_ACCELERATION} is without gravity but a combination of hardware and software sensor. Where as \texttt{TYPE\_ACCELEROMETER} comes the measurements only from hardware. But there have been some bias removal from the sensor such bias from different temperature. \cite[]{android:sensorEvent}\\
It would be a reasonable assumption that JavaScripts acceleration without gravity gets sensor data from Androids \texttt{TYPE\_ACCELEROMETER} and JavaScripts acceleration including gravity gets data from \texttt{TYPE\_LINEAR\_ACCELERATION}. Thus software calibrations or calculations have been done on the output event from the acceleration including gravity. This however is not anything that is public in any specifications such as W3C or Mozilla.~\cite[]{sensor:W3Cspec}~\cite[]{sensor:accIncludingGravity}\\ 
As a result of measurements I the used measurement of the accelerometer in measurements II is the one including gravity. \\
\\
As seen in the figures~\ref{fig:x50days}, ~\ref{fig:y50days} and ~\ref{fig:z50days} the \textit{Google Nexus 7} has not changed much over the 50 days compared to the \textit{Sony Xperia Z1 Compact} that especially has changed on the y-axis. The reason for the difference of accelerometer change over time may be due to the \textit{Google Nexus 7} being in the same place during those 50 days, and therefore only used when the tests were performed. Unlike the \textit{Sony Xperia} device that was used daily and might have been dropped at some point. An additional fact about the measurements is that both devices have changed its OS between measurements 2 and 3, from Android version 4.4.4 to 5.1.1 and that different browser were used (Opera, Chrome and Firefox). Only the \textit{Sony Xperia} device had changed and not the \textit{Google Nexus}, which leads to the conclusion that OS version or browser does not matter noticeable and that the use of the device may affect the accelerometer.\\
\\
When comparing distances between the time features there are some values to discuss. The percentage that is calculated in table~\ref{tab:addlabel} is the percentage calculated to compare if the the distances of features between all 60 devices is larger than the distances between measurements of one device. If the min-distance has a percentage more than 100\% that means that there are different devices that have closer feature-distance than the ones between one device, thus not a good candidate for fingerprinting. Average deviation, kurtosis and skewness were excluded from the table since their percentage were all too high (the min-distance in percent were higher than 100\%). The median distance of the features gives a value of the normal case of the measurements. For example the median mean-distance between all devices is ten times longer than the median mean-distance between the measurements of \textit{Nexus7}. The lower percentage the lower risk of that another device has similar values. From this point of view the Mean, Maximum, Minimum and Median makes the best features of fingerprinting.

\subsection{Method}
As discussed in the beginning of the section above the JavaScript or Android/iOS doing some calibration with the sensor that effects the result if not dealing with raw data. But as also mentioned is the data used in measurements II probably as raw data as you can get without reading from the accelerometer alone. To read directly from the accelerometer would however been hard since manual calibration of noise caused be temperature had to be done.

\section{Gyroscope}
Discussion about the result of the gyroscope measurements and the method used to get gyroscope data from the mobile device.
\subsection{Result}
The first method used to compare the measurements was based on research of the accelerometer since there were no earlier research on the gyroscope. This may affect the outcome since there may be other features that would have given better result. \\
The other method where calculating the Allan variance that is used for calibration of gyroscope, did not give the expected result. Since the variance is used for gyroscope calibration it may be the case that it already is calculated and compensated for in the device.\\
\\
The gyroscope seems much more sensitive in measurements than the accelerometer and therefore it is harder to extract the constant bias. The fact that Android or JavaScript does not reveal information on what bias compensation that has been done before the developer get the measurement data, which makes that part harder. That the gyroscope is much more sensitive than the accelerometer can be seen when reading from table~\ref{tab:featureGyro} were the \textit{Sony Xperia Z1 Compact} device changed the min median distance with 75\% and the \textit{Google Nexus 7} only with 6\%. The \textit{Sony Xperia} has been used over the fifty days of measurements, compared to the \textit{Nexus} that were only used at the time of the measurements. \\
\\
A thing to take in account before the constant noise from the gyroscope is ruled out is if the sensor data gotten from JavaScript contains software calibrations or the output data coming raw from the sensor. \\
The Android developer page about sensor events state that they make factory calibration and temperature compensation even on their uncalibrated sensor events (magnetometer and gyroscope). That is relativity new feature added in Android 4.3 Jelly Bean (API level 18~\cite{android:API18}) from 2013 but the original once used since Android 1.5 Cupcake (API level 3~\cite{android:API3}) from 2009 makes some more noise compensation and calibration. What kind of compensation and calibration done is not public.~\cite[]{android:sensorEvent}\\
Since the output of both the calibrated and uncalibrated sensor is in rad/s implies that it could be some software calibration in the date, not knowing where it is done.

\subsection{Method}
The method using JavaScripts listener to collect the data seems to work as expected. The question to ask is the same as for the accelerometer how much calibration and compensation of bias and drifts already done before the software developers get the output from the gyroscope. The positive thing about using JavaScript instead of developing an application is that the diversity of the collected devices is much better. It also gets easier to collect measurements since it is a web-page is much easier to spread and no installation is needed, in context to an application that has to be installed. The gain of using an Android application when measuring the gyroscope would be that Android provides an uncalibrated version of the gyroscope since 2013~\cite{android:API18}. This rawer data may result in better feature values in time domain or Allan variance.


\section{Camera}
This section discusses the result and method used for evaluating the camera sensor as a fingerprinting characteristic.
\subsection{Result}
The result of the camera sensor were not as good as expected or as good as in the research by ~\cite[]{sensor:camera:DCIdent} were PRNU also was used. The significant differences is the use of a mobile device camera instead of a digital camera. Although the high level specification seems to be comparable with the digital cameras from 2009, since they had around 11 mega-pixels, an images size of around 4000x3000 pixels, and digital zoom of 4 times and had HD video recording width 30 fps.~\cite[]{gbg:kamera} This is about the average smart-phones camera today, but some other specifications may have other impact as ISO, optical zoom etc.

\subsection{Method}
The two methods used for collecting picture features had different advantages and disadvantages. The video-collecting done in connection to the second measurement were good in terms of measurability since it was easy to record a video of five seconds and just send. It generated 100-200 which also made an enough pictures for a trait. On the other hand that lead to worse result in terms of uniqueness. \\
The second way of collecting data was not as good in terms of measurability but it got slightly more uniqueness but far from good enough.


\section{The work in a wider context}\label{sec:ethical}
There is a lot of issues to discuss in terms of privacy and integrity when dealing with the sensor of the device. To begin with neither of the motion sensors require any permission to read when visiting a web-page. If there is an easy way of identifying a device by a sensor the days of using cookies will be long gone. Which of course can have an advantage in terms of user-ability, but as valuable your personal information is today for the commercial and advertising it’s hard to set a value for something that could identify you everywhere on the Internet. The tracking possibilities are enormous and have to be concerned if this type of identifying can be done. \\
\\
There are of course some good things in the view of ethical and societal aspects. If the sensor-data is used as aimed in this thesis it gain privacy and integrity since the provided possibility of more secure authentication both between human and machine and M2M. Because, you want to know that it really is one of your heat sensors that send signals to your thermostat, or that only your mobile device that can unlock the front door.\\
\\
The point here is that fingerprinting features of a device should be treated in the same way as your biometric trait. This means that you want to have control over were your biometric trait is used. Most of us think that it is legit that Authority would use our fingerprint if it resulted in a more secure society. On the other hand most of us do not want our fingerprints to be used in commercial purposes. \\
The concept should be considered when fingerprinting a device as well. The accelerometer data may be applicable to use by banks, to your door or car. But you may not want as a login feature to a commercial site that may sell that information.