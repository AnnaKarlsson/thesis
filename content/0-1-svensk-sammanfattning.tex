Antalet enheter som är anslutna till internet växer snabbt. När man talar om enheter så menar man också de som inte har någon kontakt med människor, ex. en uppkopplad temperaturgivare till en termostat. Dessa typer av enheter är de som förväntas växa mest, vilket är anledningen till att området för att unikt identifiera dessa enheter kräver mer undersökning. Det här examensarbetet inkluderar mätningar och utvärdering på användningen av sensorerna accelerometer, kamera och gyroskope på mobiltelefoner för att undersöka i vilken utsträckning de går att identifiera som unika enheter. Det kan liknas med ett fingeravtryck för mobiltelefonen. Den metod som används bygger på tidigare forskning inom sensoridentifiering tillsammans med metoder som används för att utforma ett biometriskt system. Kombinationen av långa beprövade metoder inom biometriområdet och ny forskning inom identifiering av sensorer är en nytt sätt för att titta på enheters fingeravtryck.
