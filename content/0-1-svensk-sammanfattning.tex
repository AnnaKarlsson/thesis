Antalet anslutna enheter som är anslutna till internet växer snabbt, när man talar om enheter så menar man också de som inte har någon kontakt med människor, ex. en uppkopplad temperaturgivare till en termostat. Denna typer av enheter är de som förväntas växa mest. Vilket är anledningen till att området för att unikt identifiera dessa enheter kräver mer undersökning. Det här exjobbet inkluderar mätningar och utvärdering på användningen av accelerometern, kamera och gyroskop sensorer på en mobil för att undersöka i vilken utsträcning de går att identifiera som unika enheter. Alltså som ett fingeravtryck för sensorer. Den metod som används bygger på tidigare forskning inom sensor identifiering tillsammans med metoder som används för att utforma ett biometriskt system. Kombinationen med långa beprövade metoder inom biometriområdet med ny forskning inom identifiering av sensorer är en nytt sätt för att titta på enheters fingeravtryck.
