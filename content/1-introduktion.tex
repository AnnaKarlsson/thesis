\chapter{INTRODUCTION}\label{cha:intro}
This paper is the report for my master thesis in Computer Science and the last part of my education for become an engineer of information-technology in the field of secure systems. The thesis was performed on Cybercom AB in Linköping. \\
This introduction chapter will give an overview of the thesis that includes background, aims and objectives for the presented work of the thesis. 

\section{Background}\label{sec:bg}
Cars, locks, birds, stoves, refrigerator, coffee maker, watches, cat feeder, sewing machines\dots, the world of connected devices is growing rapidly. This world is known under the term `Internet of Things'. For making this things connect to each other we need secure authentication methods for knowing that things are connecting to the things they are suppose to and not anything or anyone else. \\
\\
For us humans it has become an everyday thing to using two factor authentication when accessing buildings, parts of networks, our bank and so on. When talking about two factor authentication we usually use a combination of either three things; something you \textit{know} like passwords, something you \textit{have} like tag, passport, card or phone or something you \textit{are} like iris or fingerprint.
(More about those in chapter~\ref{cha:auth}.)
Something you know or have is things that can be copied, stolen or modified fairly easy and without know all that much about the person or thing you try to authenticate as. This compared to something you are as iris, fingerprint and DNA requires much more effort and time since you can only focus on one person at a time. Machines or devices don't have those attributes as us human, they are build on hardware parts.\\ 
The background for this thesis is to explore the possibility for a machine to have a fingerprint that can be used to more securely authenticate them. This can be applied in several areas for example in the new smart homes where fridges, stoves, coffee makers and doors should communicate with each other. Another example could be when you only want to limit the access to your bank account to your phone only to avoid that an malicious user accessing your account.

\section{\textbf{ToUpdate! }Aims \& Objectives}\label{sec:aim}
Today most of the solutions for M2M authentication involves a certificate, token, UUID etc., from my opinion is this something the machine know or have. The area of fingerprinting a machine has been more investigated in line with the world of connected devices that is called IoT (Internet of Things) has grown. The aim of this thesis is to look in to if the fingerprinting methods found today, can be used as something the machine \textit{are} for two factor authentication between them. The problems I'll work to solve with this thesis is:
\begin{itemize}
	\item[] Can you create a device fingerprint by using hardware characteristics in a mobile device?
	\item[] Is this fingerprint suitable for using as a second factor for authentication between devices?
\end{itemize}
The problems above state a mobile device and not a general machine, which is one of the limitations in the thesis. When stating a mobile device leaves also leads to wireless network environment. The focus is also set to an authentication process where you are able to collect a set of data from the device in a database in a enrollment phase. This means that new devices in the network has to go trough some kind of phase were collecting the unique hardware characteristics data, just like the police has to collect fingerprint from the suspect to compare with the fingerprints from the crime scene. \\
\\
The objectives of this work and can be summed up to:
\begin{itemize}
	\item[] \textbf{Explore different hardware sensors of a mobile device} \\
	Mobile devices today are equipped with a lot of sensors and since they like other hardware has some bias that may be unique enough to differ from a device of the same model. Measurements from the gyroscope-, accelerometer- and camera-sensor will be collected and valuated from the view as fingerprints.
	\item[] \textbf{Combining M2M, two factor and biometric authentication} \\
	Biometric authentication has ways of measure and compare fingerprints, this measurements and methods will be used to evaluate and design the fingerprints of the device.
\end{itemize}

\section{\textbf{ToUpdate! }Thesis Outline}\label{sec:outline}
This introduction chapter including background, aims and objectives will give a quick view of what the thesis is about. The chapters that following is divided in different parts that maps to the different objectives listed above.
\begin{itemize}
	\item[Ch.1:]	This will give an introduction to the work done in the thesis and motivation for doing it.
	\item[Ch.2:]	How authentication is made today between machines, two factor and in biometric.
	\item[Ch.3:]	The different unique hardware characteristics of a mobile device that has been found today and how they are collected.
	\item[Ch.4:]	The measurements methods used to collect sensordata from the mobile devices.
	\item[Ch.5:]	Result of the measurements togather with a simulated authentication to test the result. 
	\item[Ch.6:]	Demo
	\item[Ch.6:]	Conclusion will except conclusions also include ethical aspects and ...further work. 
\end{itemize}